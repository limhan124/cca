

\emph{\textbf{(a)}} \texttt{Memcached} with \texttt{CPU} interference saturates at around 25K.
The tail latency is about 4x - 6x both under normal QPS and under the saturate point compared with no interference.
This result means \texttt{memcached} is computation-intensive and uses CPU heavily. When there is interference at the CPU level,
the performance drops down significantly.


\emph{\textbf{(b)}}
\texttt{Memcached} with \texttt{l1i} interference saturates at around 25K.
The tail latency is about 4x - 6x both under normal QPS and under the saturate point compared with no interference.
We expect that most applications will be to a large extent sensitive to the interference of \texttt{l1i} because every program needs to fetch instructions from \texttt{l1i}, as is also the case of \texttt{memcached}.  
From the plot, we can also see that the CPU and \texttt{l1i} are closely coupled with each other at a higher QPS. At a lower QPS, however, 
the tail latency of interference from the  CPU is  higher than the interference from \texttt{l1i}. This phenomenon is probably because the
 instructions' computation time bottlenecks the application rather than the stall time of fetching instructions at this stage.


\emph{\textbf{(c)}}
The result of interference from \texttt{l1d} and \texttt{l2} is similar. They both saturate at around 50K, and the tail latency is around 2ms at the saturated point, almost identical to the no interference case.
This result means that the interference from \texttt{l1d} and \texttt{l2} does not affect \texttt{memcached}. That is to say, most of the memory requests from \texttt{memcached} do not go to \texttt{l1d} or \texttt{l2}.
Probably this phenomenon is  because \texttt{memcached} has a working set size larger than \texttt{l2}, or its access pattern is random.


\emph{\textbf{(d)}}
The result of interference from \texttt{llc} is interesting. It saturates at around 30K. The tail latency is around 2x - 4x compared with no interference.
We can see that the tail latency increases dramatically between 10K and 20K but fluctuates between 20K and 30K. This result is probably
because the working set size at QPS 10K to 20K is smaller than \texttt{llc}, so the interference at the \texttt{llc} level will affect significantly during this stage.
However, when the QPS goes to around 20K to 30K, some memory requests will miss in \texttt{llc} and start to go to RAM, so the interference at \texttt{llc} does not affect that much.


\emph{\textbf{(e)}}
The result of interference from \texttt{memBW} is also noteworthy. 
It saturates at around 45K. 
When QPS is lower than 20K, the tail latency is almost the same as with no interference.
This phenomenon is because almost all of the memory requests at this stage go to \texttt{llc}, so the interference at the RAM level does not affect that much.
When it goes to a higher QPS, the tail latency is about 1x - 1.5x as of the case with no interference. This result is because some of the memory requests start to go to the RAM, so the interference in the RAM begins to play a role.
Also, note that  the tail latency with interference from \texttt{memBW} is much smaller than that of  the interference from \texttt{llc}.
This observation means that, even with higher QPS, most of the memory requests go to \texttt{llc}.   So \texttt{memcached} is more sensitive to the interference from \texttt{llc}.
